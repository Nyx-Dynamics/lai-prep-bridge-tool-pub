\documentclass[11pt]{article}
\usepackage[margin=0.75in]{geometry}
\usepackage{helvet}
\renewcommand{\familydefault}{\sfdefault}
\usepackage{xcolor}
\usepackage{amssymb}
\usepackage{titlesec}
\usepackage{array}
\usepackage{tabularx}
\usepackage{hyperref}

\titleformat{\section}{\Large\bfseries\color{blue!70!black}}{\thesection}{1em}{}
\titleformat{\subsection}{\large\bfseries\color{blue!50!black}}{\thesubsection}{1em}{}

\begin{document}

\begin{center}
{\Huge\bfseries Supplementary File S1}\\[0.3cm]
{\LARGE LAI-PrEP Bridge Period Quick Reference Card}\\[0.2cm]
{\large One-Page Clinical Decision Guide}\\[0.5cm]
{\normalsize For Point-of-Care Use}\\[0.2cm]
{\small \textit{Version 2.1 | October 2025}}
\end{center}

\vspace{0.5cm}

\section*{THE PROBLEM}

\begin{center}
\colorbox{red!20}{\parbox{0.9\textwidth}{\centering\large\bfseries
47\% of prescribed patients never receive their first injection}}
\end{center}

\section*{STEP 1: ASSESS RISK}

\subsection*{Is patient currently on oral PrEP?}
\begin{itemize}
\item \textbf{YES} $\rightarrow$ SUCCESS RATE: 85-90\% | \textbf{ACTION: Priority for rapid transition}
\item \textbf{NO} $\rightarrow$ Continue to population assessment
\end{itemize}

\subsection*{Population Category \& Baseline Success Rates}

\begin{center}
\begin{tabular}{|l|c|c|}
\hline
\textbf{Population} & \textbf{Success Rate} & \textbf{Risk Level} \\
\hline
MSM & 55\% & Moderate \\
Cisgender women & 45\% & Moderate-High \\
Transgender women & 50\% & Moderate \\
\textbf{Adolescents (16-24)} & \textbf{35\%} & \textbf{HIGH} \\
\textbf{PWID} & \textbf{25\%} & \textbf{VERY HIGH} \\
\hline
\end{tabular}
\end{center}

\subsection*{Count Barriers (check all that apply)}

Each barrier reduces success rate (multiplicative combination):

\begin{itemize}\itemsep0em
\item[$\square$] Transportation
\item[$\square$] Childcare
\item[$\square$] Housing unstable
\item[$\square$] Insurance delays
\item[$\square$] Medical mistrust
\item[$\square$] Privacy concerns
\item[$\square$] Legal concerns
\item[$\square$] No government ID
\end{itemize}

\section*{STEP 2: SELECT INTERVENTIONS}

\subsection*{Priority One}

\textbf{If on oral PrEP + recent HIV test ($<$7 days):}
\begin{itemize}
\item \textbf{SAME-DAY SWITCHING} (+35 points)
\item Can inject today or within 3 days
\item Eliminates bridge period entirely
\end{itemize}

\textbf{If on oral PrEP (no recent test):}
\begin{itemize}
\item \textbf{RAPID ORAL-TO-INJECTABLE TRANSITION} (+35 points)
\item Schedule HIV test + injection for same week
\item 88-90\% success vs.\ 53\% for new patients
\end{itemize}

\textbf{If PWID:}
\begin{itemize}
\item \textbf{HARM REDUCTION INTEGRATION} (+25-35 points)
\item Partner with syringe service program
\item ESSENTIAL - traditional clinic will fail
\end{itemize}

\subsection*{PRIORITY (Implement for At-Risk Patients)}

\begin{center}
\small
\begin{tabular}{|l|c|l|}
\hline
\textbf{Intervention} & \textbf{Impact} & \textbf{Best For} \\
\hline
Patient Navigation & +12-20 pts & Adolescents, women, anyone $<$50\% \\
Peer Navigation & +15-20 pts & PWID, transgender, complex barriers \\
Accelerated Testing & +15-20 pts & All new patients \\
Transportation Support & +10-15 pts & When barrier identified \\
Childcare Support & +8-12 pts & Parents \\
Expedited Insurance & +12-15 pts & When delays expected \\
\hline
\end{tabular}
\end{center}

\subsection*{PRIORITY}
\begin{itemize}\itemsep0em
\item Text message reminders: +10-15 points
\item Telehealth counseling: +10-15 points
\item Mobile delivery: +15-25 points (if available)
\end{itemize}

\section*{STEP 3: CALCULATE FINAL SUCCESS RATE}

\begin{enumerate}
\item Start with baseline (population rate)
\item Apply barrier impact (multiplicative combination)
\item Select 3-5 evidence-based interventions addressing diverse mechanisms. \textbf{Combined effect calculation:} Algorithm applies 70\% diminishing returns factor to sum of intervention effects (reflecting overlapping mechanisms and patient saturation), then caps final success at 95\% maximum. This ceiling varies by baseline: low-baseline patients can improve more than high-baseline patients.
\end{enumerate}

\textbf{Success Rate Interpretation:}
\begin{itemize}\itemsep0em
\item \textbf{$>$70\%}: Excellent - standard protocols OK
\item \textbf{50-69\%}: Good - navigation recommended
\item \textbf{30-49\%}: Concerning - multiple interventions needed
\item \textbf{$<$30\%}: Critical - intensive support required
\end{itemize}

\textit{Note: See Supplementary File S7 for complete mathematical formulas and two-stage model details.}

\section*{STEP 4: IMPLEMENT \& TRACK}

\textbf{What to Do (at prescription visit):}
\begin{enumerate}\itemsep0em
\item[$\square$] Assign navigator (if high risk)
\item[$\square$] Order HIV test (expedited processing)
\item[$\square$] Provide transportation voucher (if needed)
\item[$\square$] Schedule injection appointment
\item[$\square$] Submit insurance authorization
\item[$\square$] Set up text reminders
\item[$\square$] Document barriers in chart
\end{enumerate}

\textbf{Follow-Up Timeline:}
\begin{itemize}\itemsep0em
\item \textbf{24 hours}: Navigator contacts patient
\item \textbf{48 hours}: Text reminder before HIV test
\item \textbf{7-14 days}: Target for first injection (oral PrEP transitions)
\item \textbf{14-28 days}: Target for first injection (new patients)
\end{itemize}

\section*{QUICK DECISION MATRIX}

\begin{center}
\footnotesize
\begin{tabular}{|p{3.5cm}|p{3.5cm}|p{2.5cm}|}
\hline
\textbf{Patient Type} & \textbf{First Action} & \textbf{Expected Success} \\
\hline
Oral PrEP + recent test & Same-day inject & 90\% \\
Oral PrEP (any) & Rapid transition & 88-90\% \\
New MSM, minimal barriers & Standard + navigation & 60-70\% \\
New woman, 2-3 barriers & Navigation + support & 50-60\% \\
Adolescent, multiple barriers & Intensive navigation & 30-45\% \\
\textbf{PWID, traditional clinic} & \textbf{WILL FAIL} & \textbf{$<$10\%} \\
PWID, harm reduction & SSP + peer nav & 35-45\% \\
\hline
\end{tabular}
\end{center}

\section*{RED FLAGS (System Failure Likely)}

\textbf{Don't just prescribe if you see:}
\begin{itemize}
\item PWID without harm reduction partnership
\item Adolescent without navigation support
\item Multiple barriers without intervention plan
\item Insurance issues without expedited process
\end{itemize}

\textbf{These patients will NOT initiate without proactive intervention!}

\section*{KEY TAKEAWAY}

\begin{center}
\colorbox{yellow!30}{\parbox{0.9\textwidth}{\centering
\textbf{The bridge period is where we lose patients.}\\
\textbf{Proactive intervention prevents attrition.}\\
\textbf{Oral PrEP patients are your easiest wins - prioritize them!}
}}
\end{center}

\vspace{0.5cm}

\textit{Evidence Base:} HPTN 083 (4,566 MSM), HPTN 084 (3,224 women), PURPOSE-1/2 (7,521 participants), real-world implementation studies. Computational validation at 21.2M patient scale.

\textit{Configuration:} v3.1 | \textit{Zenodo DOI:} 10.5281/zenodo.17727117

\textit{Reference:} Demidont (2025). Computational Validation of LAI-PrEP Bridge Decision Support Tool. \textit{Viruses}.

\textit{For complete methodology:} See Supplementary File S7 (Intervention Library with mathematical formulas)

\end{document}
