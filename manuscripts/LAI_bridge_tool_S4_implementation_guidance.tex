\documentclass[11pt]{article}
\usepackage[margin=1in]{geometry}
\usepackage{helvet}
\renewcommand{\familydefault}{\sfdefault}
\usepackage{xcolor}
\usepackage{titlesec}
\usepackage{enumitem}
\usepackage{hyperref}

\titleformat{\section}{\Large\bfseries\color{blue!70!black}}{\thesection}{1em}{}
\titleformat{\subsection}{\large\bfseries\color{blue!50!black}}{\thesubsection}{1em}{}

\begin{document}

\begin{center}
{\Huge\bfseries Supplementary File S4}\\[0.3cm]
{\LARGE Implementation Guide for LAI-PrEP Bridge Period Decision Support Tool}\\[0.5cm]
{\large Computational Validation of Clinical Decision Support Algorithm}\\
{\large for Long-Acting Injectable PrEP Bridge Period Navigation}\\[0.3cm]
{\normalsize A.C Demidont, DO}\\[0.2cm]
{\small\textit{Viruses} Journal Supplementary Materials}
\end{center}

\vspace{0.5cm}

\section*{Purpose of This Guide}

This implementation guide provides preliminary protocols and recommendations for prospective validation of the LAI-PrEP Bridge Period Decision Support Tool. \textbf{Important caveats:}

\begin{itemize}
\item These materials are \textbf{preliminary} and require refinement through actual implementation experience
\item Computational validation establishes algorithmic precision, not clinical readiness for unrestricted deployment
\item All recommendations should be adapted to local context, resources, and patient populations
\item Prospective validation with real patient outcomes is essential before widespread adoption
\item Implementation teams should document both successes and failures to enable continuous learning
\end{itemize}

\section{Staged Implementation Pathway}

Based on our computational validation findings and critical assessment of AI suitability, we propose a structured implementation pathway balancing innovation urgency with patient safety.

\subsection{Phase 1: Immediate Actions (0--6 months)}

\subsubsection{Pilot Site Selection}

Identify \textbf{2--3 diverse clinical settings} representing populations with different baseline success rates and barrier profiles:

\begin{enumerate}[leftmargin=*]
\item \textbf{Suggested site types:}
\begin{itemize}
\item Urban academic center serving men who have sex with men (MSM) --- typically highest baseline success rates but potential insurance/authorization barriers
\item Community clinic serving cisgender women in high-prevalence area --- moderate baseline rates with transportation, childcare, and structural barriers
\item Harm reduction program serving people who inject drugs (PWID) --- lowest baseline rates with substance use, stigma, and housing instability barriers
\end{itemize}

\item \textbf{Site selection criteria:}
\begin{itemize}
\item LAI-PrEP prescribing volume: minimum 20--30 patients/year anticipated
\item Institutional commitment to systematic outcome tracking
\item Availability of patient navigation or case management resources
\item Diverse patient population for subgroup analyses
\item Electronic health record capabilities for data collection
\item Research infrastructure or quality improvement expertise
\end{itemize}

\item \textbf{Sample size considerations:}
\begin{itemize}
\item Target 50--100 patients per site over 6--12 months
\item Minimum 150 total patients across all pilot sites for adequate statistical power
\item Oversample populations with lower baseline success rates (women, PWID, adolescents) to enable equity analyses
\end{itemize}
\end{enumerate}

\subsubsection{Institutional Review and Adaptation}

Local implementation teams should critically review the external configuration file (\texttt{lai\_prep\_config.json}):

\begin{enumerate}[leftmargin=*]
\item \textbf{Barrier prevalence assessment:}
\begin{itemize}
\item Do local barrier prevalence rates match national estimates in the configuration?
\item Example: If local MSM population has 15\% insurance barriers (vs.\ 28\% national), adjust \texttt{insurance\_auth\_barrier\_rate}
\item Document all parameter modifications with justification
\end{itemize}

\item \textbf{Intervention availability audit:}
\begin{itemize}
\item Which of the 21 interventions are actually available locally?
\item Example: If expedited HIV testing not available, set \texttt{effect\_size} to 0 for that intervention
\item Identify locally-available interventions not in the library (candidates for addition)
\end{itemize}

\item \textbf{Effect size calibration:}
\begin{itemize}
\item Are national effect size estimates appropriate for local context?
\item Example: Patient navigation may have different effectiveness in rural vs.\ urban settings
\item Consider local pilot data if available; otherwise use default values initially
\end{itemize}

\item \textbf{Population-specific baseline rates:}
\begin{itemize}
\item Review baseline bridge period success rates for each population
\item Adjust if local data suggest different rates (e.g., high-performing clinic with strong navigation infrastructure)
\end{itemize}
\end{enumerate}

\subsubsection{Clinician Training}

Train providers to use the tool appropriately while maintaining clinical judgment:

\begin{enumerate}[leftmargin=*]
\item \textbf{Core competencies:}
\begin{itemize}
\item Interpret algorithmic output: understand risk scores, barrier profiles, intervention recommendations
\item Recognize model limitations: identify when model assumptions may not fit specific patients
\item Exercise clinical override: document decisions when clinical judgment differs from algorithm
\item Provide patient-centered care: use recommendations as decision support, not replacement for human judgment
\end{itemize}

\item \textbf{Training curriculum (suggested 2--3 hour session):}
\begin{itemize}
\item \textit{Module 1 (30 min):} LAI-PrEP bridge period attrition crisis and evidence base
\item \textit{Module 2 (45 min):} Tool architecture, intervention library, mechanism diversity scoring
\item \textit{Module 3 (45 min):} Clinical case scenarios with tool output interpretation
\item \textit{Module 4 (30 min):} Documentation requirements, override protocols, feedback mechanisms
\end{itemize}

\item \textbf{Ongoing support:}
\begin{itemize}
\item Weekly case conferences reviewing challenging patients
\item Monthly calibration meetings comparing predicted vs.\ actual outcomes
\item Access to implementation science team for technical questions
\end{itemize}
\end{enumerate}

\subsubsection{Data Collection Protocols}

Establish systematic outcome tracking infrastructure:

\begin{enumerate}[leftmargin=*]
\item \textbf{Required data elements:}
\begin{itemize}
\item Patient demographics: age, gender, race/ethnicity, socioeconomic indicators
\item Population category: MSM, cisgender women, PWID, adolescents (may overlap)
\item Baseline assessment: date of LAI-PrEP prescription, identified barriers, risk factors
\item Algorithmic output: predicted success probability, recommended interventions, mechanism diversity score
\item Interventions delivered: which recommendations implemented, fidelity assessment, timing
\item Primary outcome: bridge period success (received first injection within 60 days) vs.\ attrition
\item Secondary outcomes: time to first injection, barriers encountered, reasons for attrition
\item Clinical overrides: instances where provider deviated from recommendations with justification
\end{itemize}

\item \textbf{Data quality assurance:}
\begin{itemize}
\item Real-time data entry with validation rules
\item Monthly audits for completeness and accuracy
\item Standardized definitions for barriers and interventions
\item Regular calibration across sites to ensure consistent measurement
\end{itemize}

\item \textbf{Ethical considerations:}
\begin{itemize}
\item Institutional review board approval for research (if applicable)
\item Quality improvement exemption (if applicable)
\item Patient consent for data use
\item Data privacy and security protocols (HIPAA compliance)
\end{itemize}
\end{enumerate}

\subsection{Phase 2: Pilot Validation (6--12 months)}

Conduct rigorous evaluation of algorithm performance with real patients:

\subsubsection{Outcome Analysis}

Collect and analyze \textbf{150--300 patient outcomes} across pilot sites:

\begin{enumerate}[leftmargin=*]
\item \textbf{Calibration assessment:}
\begin{itemize}
\item \textit{Overall calibration:} Do predicted success rates match actual rates?
\item \textit{Subgroup calibration:} Evaluate separately for MSM, women, PWID, adolescents
\item \textit{Risk stratification:} Compare outcomes across predicted risk quartiles
\item \textit{Statistical tests:} Hosmer-Lemeshow goodness-of-fit, calibration plots
\end{itemize}

\item \textbf{Discrimination evaluation:}
\begin{itemize}
\item Does the tool effectively separate high-risk from low-risk patients?
\item Calculate area under the receiver operating characteristic curve (AUROC)
\item Assess sensitivity and specificity at different risk thresholds
\item Evaluate positive and negative predictive values
\end{itemize}

\item \textbf{Intervention effectiveness:}
\begin{itemize}
\item Do recommended interventions achieve predicted improvements?
\item Compare outcomes: interventions received vs.\ not received
\item Assess dose-response: more interventions → better outcomes?
\item Evaluate mechanism diversity: Do diverse mechanisms improve outcomes beyond number of interventions?
\end{itemize}

\item \textbf{Failure mode identification:}
\begin{itemize}
\item What patient characteristics result in inaccurate predictions?
\item Identify barriers not adequately captured in the model
\item Document intervention implementation challenges
\item Analyze clinical override patterns (when and why providers deviated)
\end{itemize}
\end{enumerate}

\subsubsection{Equity Analysis}

Evaluate algorithmic fairness across populations:

\begin{enumerate}[leftmargin=*]
\item \textbf{Differential calibration:} Does accuracy vary by race, ethnicity, socioeconomic status?
\item \textbf{Differential benefit:} Do interventions work equally well across subgroups?
\item \textbf{Access equity:} Are recommended interventions equally available to all populations?
\item \textbf{Outcome disparities:} Does tool use narrow or widen existing gaps in bridge period success?
\end{enumerate}

\subsubsection{Parameter Refinement}

Based on pilot data, update configuration parameters:

\begin{enumerate}[leftmargin=*]
\item Revise barrier prevalence estimates where local rates differ from national
\item Adjust intervention effect sizes based on observed outcomes
\item Update population-specific baseline rates if needed
\item Add new barriers or interventions identified during pilot
\item Modify mechanism diversity scoring if redundancies observed
\end{enumerate}

\subsection{Phase 3: Multi-Site Expansion (12--24 months)}

Scale to broader geographic and demographic diversity:

\begin{enumerate}[leftmargin=*]
\item \textbf{Site expansion:}
\begin{itemize}
\item Expand to \textbf{10--15 sites} representing diverse geographies, healthcare systems, and patient populations
\item Include mix of academic medical centers, community health centers, harm reduction programs, and telehealth providers
\item Target sites in high-HIV-burden regions (Southern U.S., urban epicenters, rural areas)
\end{itemize}

\item \textbf{Sample size goals:}
\begin{itemize}
\item Collect \textbf{500--1,000 patient outcomes} across all sites
\item Ensure adequate representation: minimum 100 patients each for MSM, women, PWID, adolescents
\item Oversample underrepresented populations for subgroup analyses
\end{itemize}

\item \textbf{Equity analyses:}
\begin{itemize}
\item Comprehensive calibration evaluation across race, ethnicity, gender, age, socioeconomic status
\item Assess intervention access and effectiveness disparities
\item Document health system adaptations needed to ensure equitable care
\item Identify populations or settings where algorithm performs poorly
\end{itemize}

\item \textbf{Dissemination:}
\begin{itemize}
\item Publish validation results in peer-reviewed implementation science journal
\item Present findings at scientific conferences (CROI, IAS, USCA)
\item Release updated configuration files with refined parameters
\item Develop implementation toolkit for other sites
\end{itemize}
\end{enumerate}

\subsection{Phase 4: Continuous Quality Improvement (24+ months)}

Establish sustainable infrastructure for ongoing monitoring and refinement:

\begin{enumerate}[leftmargin=*]
\item \textbf{Automated feedback loops:}
\begin{itemize}
\item Real-time comparison of predicted vs.\ actual outcomes
\item Dashboards displaying calibration metrics over time
\item Alert systems flagging performance degradation
\item Regular reports to clinical teams and administrators
\end{itemize}

\item \textbf{Algorithmic drift monitoring:}
\begin{itemize}
\item Track changes in calibration and discrimination over time
\item Identify when recalibration or retraining needed
\item Distinguish true drift (changing populations) from data quality issues
\item Implement version control for configuration updates
\end{itemize}

\item \textbf{Evidence updates:}
\begin{itemize}
\item Monitor emerging LAI-PrEP implementation literature
\item Incorporate new trial results (HPTN 102, HPTN 103, PURPOSE extensions)
\item Update intervention library as new strategies emerge
\item Revise effect size estimates based on accumulating evidence
\end{itemize}

\item \textbf{Implementation support:}
\begin{itemize}
\item Develop comprehensive implementation guides
\item Create training materials and webinars
\item Establish technical assistance infrastructure
\item Build community of practice among implementing sites
\end{itemize}

\item \textbf{International adaptation:}
\begin{itemize}
\item Create frameworks for adapting tool to diverse healthcare systems
\item Conduct validation studies in low- and middle-income countries
\item Address context-specific barriers (e.g., patent medicine vendors in West Africa)
\item Ensure cultural appropriateness of interventions
\end{itemize}
\end{enumerate}

\section{Research Priorities}

Advancing both the science and practice of bridge period management requires coordinated research across multiple domains:

\subsection{Methodological Innovations}

\begin{enumerate}[leftmargin=*]
\item \textbf{Synergistic barrier interactions:}
\begin{itemize}
\item Current model assumes additive effects; develop methods for modeling synergies
\item Example: Transportation barriers may interact multiplicatively with childcare barriers
\item Use machine learning to detect interaction patterns in large datasets
\item Validate interaction models prospectively
\end{itemize}

\item \textbf{Time-to-event modeling:}
\begin{itemize}
\item Predict not just initiation success, but timing of first injection
\item Survival analysis methods accounting for right-censoring
\item Identify intervention effects on time-to-initiation
\item Enable resource allocation optimization (intensive early support vs.\ extended follow-up)
\end{itemize}

\item \textbf{Unmeasured barrier detection:}
\begin{itemize}
\item Machine learning approaches identifying latent barriers through pattern recognition
\item Cluster analysis grouping patients with similar failure modes
\item Natural language processing of clinical notes to identify novel barriers
\item Semi-supervised learning when labeled training data limited
\end{itemize}

\item \textbf{Causal inference methods:}
\begin{itemize}
\item Distinguish true intervention effects from selection bias
\item Instrumental variable approaches when randomization infeasible
\item Propensity score matching for observational comparisons
\item Difference-in-differences for policy evaluations
\end{itemize}
\end{enumerate}

\subsection{Evidence Generation}

\begin{enumerate}[leftmargin=*]
\item \textbf{LAI-PrEP-specific trials:}
\begin{itemize}
\item Complete HPTN 102 (women) and HPTN 103 (PWID)
\item PURPOSE-3 examining transgender populations
\item Adolescent LAI-PrEP implementation trials
\item Comparative effectiveness: CAB vs.\ lenacapavir bridge periods
\end{itemize}

\item \textbf{Implementation trials:}
\begin{itemize}
\item Randomized comparisons: systematic barrier-focused navigation vs.\ standard care
\item Stepped-wedge designs for pragmatic evaluation
\item Effectiveness-implementation hybrid designs
\item Cost-effectiveness analyses
\end{itemize}

\item \textbf{Cultural adaptation research:}
\begin{itemize}
\item Intervention adaptations for diverse international settings
\item Community-engaged participatory research methods
\item Qualitative studies identifying culturally-specific barriers
\item Validation studies in sub-Saharan Africa, Asia-Pacific, Latin America
\end{itemize}

\item \textbf{Intersectionality research:}
\begin{itemize}
\item Intersection of multiple marginalizations: race, poverty, criminalization, gender identity
\item Structural violence and bridge period outcomes
\item Policy interventions addressing root causes of disparities
\item Community-level interventions beyond individual patient support
\end{itemize}
\end{enumerate}

\subsection{Implementation Science}

\begin{enumerate}[leftmargin=*]
\item \textbf{Fidelity measurement:}
\begin{itemize}
\item Develop validated scales assessing intervention implementation quality
\item Distinguish fidelity (adherence to protocol) from adaptation (appropriate modifications)
\item Link fidelity to patient outcomes
\item Create audit tools for quality assurance
\end{itemize}

\item \textbf{Sustainability models:}
\begin{itemize}
\item Financing mechanisms ensuring navigation programs persist beyond pilot funding
\item Integration into routine clinical workflows
\item Workforce development and training pipelines
\item Policy advocacy for reimbursement of navigation services
\end{itemize}

\item \textbf{Healthcare system adaptations:}
\begin{itemize}
\item Organizational changes required for routine bridge period management
\item Electronic health record integration requirements
\item Referral networks and care coordination structures
\item Performance metrics and quality indicators
\end{itemize}

\item \textbf{Policy research:}
\begin{itemize}
\item Insurance authorization streamlining
\item Expedited HIV testing protocols
\item Pharmacy dispensing models
\item Same-day initiation feasibility
\item Task-shifting to nurses, pharmacists, community health workers
\end{itemize}
\end{enumerate}

\section{Quality Assurance Framework}

\subsection{Performance Monitoring Metrics}

Sites should track the following metrics monthly:

\begin{enumerate}[leftmargin=*]
\item \textbf{Outcome metrics:}
\begin{itemize}
\item Bridge period success rate (overall and by population)
\item Time from prescription to first injection (median, IQR)
\item Attrition rate and reasons for attrition
\item 6-month and 12-month persistence on LAI-PrEP
\end{itemize}

\item \textbf{Algorithmic performance:}
\begin{itemize}
\item Calibration: predicted vs.\ actual success rates
\item Discrimination: AUROC for risk stratification
\item Subgroup performance: calibration within populations
\item Override rate: frequency of clinical judgment overriding recommendations
\end{itemize}

\item \textbf{Process metrics:}
\begin{itemize}
\item Intervention delivery rate: \% of recommended interventions actually provided
\item Intervention timeliness: lag between recommendation and delivery
\item Documentation completeness: \% of required data fields captured
\item Patient engagement: \% accepting navigation services
\end{itemize}

\item \textbf{Equity metrics:}
\begin{itemize}
\item Outcome disparities by race, ethnicity, gender, socioeconomic status
\item Intervention access disparities
\item Calibration equity across subgroups
\item Representation: demographics of patients served vs.\ indicated population
\end{itemize}
\end{enumerate}

\subsection{Corrective Action Triggers}

Establish thresholds for initiating corrective actions:

\begin{enumerate}[leftmargin=*]
\item \textbf{Poor overall calibration:} Predicted-observed difference $>$5 percentage points for 2 consecutive months → Parameter review and recalibration

\item \textbf{Subgroup calibration failure:} Any population with predicted-observed difference $>$10 points → Targeted parameter adjustment for that population

\item \textbf{Declining performance:} Bridge period success rate declining $>$5 points over 3 months → Process evaluation and intervention reinforcement

\item \textbf{Low intervention delivery:} $<$50\% of recommended interventions provided → Workflow redesign and barrier assessment

\item \textbf{Widening disparities:} Outcome gap between populations increasing $>$5 points → Equity-focused intervention
\end{enumerate}

\section{Limitations and Cautions}

Implementation teams should recognize these important limitations:

\begin{enumerate}[leftmargin=*]
\item \textbf{Parameter uncertainty:} Many parameters extrapolated from non-LAI-PrEP populations; prospective validation essential

\item \textbf{Local variation:} National estimates may not match local context; adaptation required

\item \textbf{Implementation challenges:} Recommended interventions require resources, training, and organizational commitment

\item \textbf{Patient heterogeneity:} Population-level predictions may not apply to individual patients; clinical judgment essential

\item \textbf{Unmeasured barriers:} Model cannot account for barriers not explicitly measured

\item \textbf{Causal assumptions:} Intervention effects assumed causal; confounding possible in observational validation

\item \textbf{Generalizability:} Validation in U.S. settings may not generalize to international contexts

\item \textbf{Evolving evidence:} LAI-PrEP implementation science rapidly evolving; continuous updates required
\end{enumerate}

\section{Conclusion}

This implementation guide provides preliminary protocols for prospective validation of the LAI-PrEP Bridge Period Decision Support Tool. Successful implementation requires:

\begin{itemize}
\item \textbf{Staged deployment} balancing innovation with safety
\item \textbf{Systematic outcome tracking} enabling evidence-based refinement
\item \textbf{Equity focus} ensuring benefits reach populations with greatest barriers
\item \textbf{Continuous learning} from both successes and failures
\item \textbf{Transparent reporting} building the evidence base for decision support in HIV prevention
\end{itemize}

These materials should be adapted to local context and refined based on implementation experience. The ultimate goal is translating computational validation into real-world impact: preventing HIV infections by systematically addressing the bridge period attrition crisis.

\vspace{1cm}

\section*{Additional Resources}

\begin{itemize}
\item \textbf{Supplementary File S1:} Clinician Quick-Reference Card
\item \textbf{Supplementary File S2:} Patient Information Handout
\item \textbf{Supplementary File S3:} Machine Readable Data Files
\item \textbf{Supplementary File S4:} Implementation Guide
\item \textbf{Supplementary File S5:} Clinical Decision Flowchart
\item \textbf{Supplementary File S6:} Non-Technical Summary
\item \textbf{Supplementary File S7:} Complete Intervention Library
\item \textbf{Supplementary File S8:} Code Repository
All source code can be located at \textbf{\textbf{Zenodo DOI}:\url{https://zenodo.org/uploads/17727117#:~:text=10.5281/zenodo.17727117}}
\item \textbf{Technical Support:} [Contact acdemidont@nyxdynamics.org]
\end{itemize}

\end{document}