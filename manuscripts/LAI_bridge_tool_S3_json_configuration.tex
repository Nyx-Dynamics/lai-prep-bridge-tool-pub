\documentclass[11pt]{article}
\usepackage[margin=1in]{geometry}
\usepackage{helvet}
\renewcommand{\familydefault}{\sfdefault}
\usepackage{xcolor}
\usepackage{titlesec}
\usepackage{enumitem}
\usepackage{hyperref}
\usepackage{listings}

\titleformat{\section}{\Large\bfseries\color{blue!70!black}}{\thesection}{1em}{}
\titleformat{\subsection}{\large\bfseries\color{blue!50!black}}{\thesubsection}{1em}{}

% Configure listings for JSON
\lstdefinelanguage{json}{
    basicstyle=\small\ttfamily,
    numbers=left,
    numberstyle=\scriptsize,
    stepnumber=1,
    numbersep=8pt,
    showstringspaces=false,
    breaklines=true,
    frame=lines,
    backgroundcolor=\color{gray!10},
    literate=
     *{0}{{{\color{blue!80!black}0}}}{1}
      {1}{{{\color{blue!80!black}1}}}{1}
      {2}{{{\color{blue!80!black}2}}}{1}
      {3}{{{\color{blue!80!black}3}}}{1}
      {4}{{{\color{blue!80!black}4}}}{1}
      {5}{{{\color{blue!80!black}5}}}{1}
      {6}{{{\color{blue!80!black}6}}}{1}
      {7}{{{\color{blue!80!black}7}}}{1}
      {8}{{{\color{blue!80!black}8}}}{1}
      {9}{{{\color{blue!80!black}9}}}{1}
      {:}{{{\color{black}{:}}}}{1}
      {,}{{{\color{black}{,}}}}{1}
      {\{}{{{\color{black}{\{}}}}{1}
      {\}}{{{\color{black}{\}}}}}{1}
      {[}{{{\color{black}{[}}}}{1}
      {]}{{{\color{black}{]}}}}{1},
}

\begin{document}

\begin{center}
{\Huge\bfseries Supplementary File S3}\\[0.3cm]
{\LARGE Machine-Readable Data Files}\\[0.5cm]
{\large Configuration and Patient Input Examples for LAI-PrEP Bridge Period Decision Support Tool}\\[0.3cm]
{\normalsize Adrian C. Demidont}\\[0.2cm]
{\small\textit{Viruses} Journal Supplementary Materials}
\end{center}

\vspace{0.5cm}

\section*{Purpose of This File}

This supplementary file provides machine-readable data files for the LAI-PrEP Bridge Period Decision Support Tool, enabling complete reproducibility and facilitating independent validation:

\begin{enumerate}
\item \textbf{Configuration File} (\texttt{lai\_prep\_config.json}): Defines algorithmic parameters, evidence-based interventions, population characteristics, and barrier impacts
\item \textbf{Patient Input Examples}: Demonstrates expected input format and provides realistic test cases for tool validation
\end{enumerate}

The configuration-driven architecture enables:

\begin{itemize}
\item \textbf{Parameter updates} without code modification
\item \textbf{Institutional adaptation} to local contexts and evidence
\item \textbf{Transparent review} of all algorithmic assumptions
\item \textbf{Reproducible research} with versioned configurations
\item \textbf{Evidence integration} as new research emerges
\end{itemize}

\section{Configuration Structure Overview}

The complete configuration file contains six major sections:

\begin{enumerate}
\item \textbf{Populations} (7 entries): Baseline attrition rates and priority interventions for each population
\item \textbf{Barriers} (13 entries): Structural barriers with quantified impacts on bridge period navigation
\item \textbf{Interventions} (21 entries): Evidence-based interventions with effect sizes and implementation details
\item \textbf{Healthcare Settings} (8 entries): Setting-specific recommendations and resource availability
\item \textbf{Risk Categories} (3 entries): Thresholds for risk stratification
\item \textbf{Algorithm Parameters}: Technical parameters for probability calculations
\end{enumerate}

\section{Representative Examples}

The following excerpts demonstrate the structure and content of each section. The complete configuration file (\texttt{lai\_prep\_config.json}) is available in the GitHub repository.

\subsection{Population Configuration Example}

Populations are defined with baseline attrition rates derived from clinical trials and real-world implementation studies:

\begin{lstlisting}[language=json,caption={Population configuration excerpt}]
"populations": {
  "MSM": {
    "name": "Men who have sex with men",
    "baseline_attrition": 0.45,
    "attrition_range": [0.40, 0.50],
    "evidence_level": "strong",
    "evidence_source": "HPTN 083 (n=4,566)",
    "clinical_notes": "MSM: Address stigma, privacy concerns, and social network disclosure. HPTN 083 showed 89% relative risk reduction.",
    "priority_interventions": [
      "PATIENT_NAVIGATION",
      "PEER_NAVIGATION",
      "SAME_DAY_SWITCHING"
    ]
  },
  "CISGENDER_WOMEN": {
    "name": "Cisgender women",
    "baseline_attrition": 0.55,
    "attrition_range": [0.50, 0.60],
    "evidence_level": "strong",
    "evidence_source": "HPTN 084 (n=3,224), PURPOSE-1 (n=5,338)",
    "clinical_notes": "Women: Address medical mistrust, structural barriers (transportation, childcare). HPTN 084 showed 89% superior efficacy; PURPOSE-1 had zero infections in 5,338 women.",
    "priority_interventions": [
      "PATIENT_NAVIGATION",
      "TRANSPORTATION_SUPPORT",
      "CHILDCARE_SUPPORT",
      "MEDICAL_MISTRUST_INTERVENTION"
    ]
  },
  "PWID": {
    "name": "People who inject drugs",
    "baseline_attrition": 0.75,
    "attrition_range": [0.70, 0.80],
    "evidence_level": "emerging",
    "evidence_source": "Oral PrEP cascade extrapolation, PURPOSE-4 (ongoing)",
    "clinical_notes": "PWID: Harm reduction approach essential. No abstinence requirement. PURPOSE-4 trial (ongoing) will provide critical implementation evidence.",
    "priority_interventions": [
      "HARM_REDUCTION_INTEGRATION",
      "PEER_NAVIGATION",
      "LOW_BARRIER_PROTOCOLS",
      "MOBILE_DELIVERY"
    ]
  }
}
\end{lstlisting}

\subsection{Barrier Configuration Example}

Barriers are quantified based on their impact on bridge period navigation success:

\begin{lstlisting}[language=json,caption={Barrier configuration excerpt}]
"barriers": {
  "TRANSPORTATION": {
    "name": "Transportation barriers",
    "impact": 0.10,
    "evidence_level": "strong",
    "affected_populations": [
      "CISGENDER_WOMEN",
      "ADOLESCENT",
      "PWID",
      "PREGNANT_LACTATING"
    ],
    "description": "Lack of reliable transportation to multiple appointments"
  },
  "INSURANCE_DELAYS": {
    "name": "Insurance authorization delays",
    "impact": 0.12,
    "evidence_level": "strong",
    "affected_populations": [
      "MSM",
      "CISGENDER_WOMEN",
      "TRANSGENDER_WOMEN",
      "ADOLESCENT",
      "GENERAL"
    ],
    "description": "Prior authorization requirements causing delays"
  },
  "HOUSING_INSTABILITY": {
    "name": "Housing instability",
    "impact": 0.15,
    "evidence_level": "strong",
    "affected_populations": ["PWID"],
    "description": "Homelessness or unstable housing affecting follow-up"
  }
}
\end{lstlisting}

\subsection{Intervention Configuration Example}

Interventions are defined with evidence-based effect sizes and implementation requirements:

\begin{lstlisting}[language=json,caption={Intervention configuration excerpt}]
"interventions": {
  "PATIENT_NAVIGATION": {
    "name": "Patient navigation services",
    "improvement": 0.15,
    "evidence_level": "strong",
    "evidence_source": "RCT meta-analysis (k=23, OR=1.85)",
    "mechanisms": ["STRUCTURAL_BARRIER_REDUCTION", "COORDINATION"],
    "target_barriers": [
      "TRANSPORTATION",
      "SCHEDULING_CONFLICTS",
      "INSURANCE_DELAYS"
    ],
    "implementation_complexity": "moderate",
    "cost_tier": "medium",
    "description": "Dedicated navigator to coordinate appointments, address barriers, and provide follow-up"
  },
  "SAME_DAY_SWITCHING": {
    "name": "Same-day switching from oral PrEP",
    "improvement": 0.25,
    "evidence_level": "strong",
    "evidence_source": "OPERA cohort (n=302), Trio Health (n=146)",
    "mechanisms": ["TEMPORAL_COMPRESSION", "ADHERENCE_PRESERVATION"],
    "target_barriers": ["HIV_TESTING_DELAYS"],
    "implementation_complexity": "low",
    "cost_tier": "low",
    "description": "For patients on oral PrEP: skip oral lead-in, test and inject same day",
    "eligibility_criteria": "Current oral PrEP use"
  },
  "HARM_REDUCTION_INTEGRATION": {
    "name": "Harm reduction service integration",
    "improvement": 0.18,
    "evidence_level": "moderate",
    "evidence_source": "PWID PrEP cascade studies, harm reduction literature",
    "mechanisms": ["STIGMA_REDUCTION", "STRUCTURAL_BARRIER_REDUCTION"],
    "target_barriers": [
      "SUBSTANCE_USE",
      "HOUSING_INSTABILITY",
      "MEDICAL_MISTRUST"
    ],
    "implementation_complexity": "high",
    "cost_tier": "medium",
    "description": "Integrate LAI-PrEP with syringe services, medication-assisted treatment, housing support"
  }
}
\end{lstlisting}

\subsection{Healthcare Setting Configuration Example}

Healthcare settings specify available interventions and recommended protocols:

\begin{lstlisting}[language=json,caption={Healthcare setting configuration excerpt}]
"healthcare_settings": {
  "COMMUNITY_HEALTH_CENTER": {
    "name": "Community Health Center",
    "available_interventions": [
      "PATIENT_NAVIGATION",
      "TRANSPORTATION_SUPPORT",
      "ACCELERATED_TESTING",
      "INSURANCE_SUPPORT",
      "FLEXIBLE_SCHEDULING"
    ],
    "typical_bridge_period_days": 21,
    "resource_level": "moderate"
  },
  "HARM_REDUCTION_PROGRAM": {
    "name": "Harm Reduction Program/Syringe Services",
    "available_interventions": [
      "HARM_REDUCTION_INTEGRATION",
      "PEER_NAVIGATION",
      "LOW_BARRIER_PROTOCOLS",
      "MOBILE_DELIVERY",
      "FLEXIBLE_SCHEDULING"
    ],
    "typical_bridge_period_days": 28,
    "resource_level": "variable"
  }
}
\end{lstlisting}

\subsection{Risk Stratification Configuration Example}

Risk categories define thresholds for clinical decision-making:

\begin{lstlisting}[language=json,caption={Risk category configuration excerpt}]
"risk_categories": {
  "HIGH_RISK": {
    "threshold": 0.60,
    "description": "Predicted success rate <40%",
    "recommended_actions": [
      "Intensive navigation (minimum 3 contacts)",
      "Multiple intervention modalities",
      "Close clinical monitoring",
      "Consider expedited protocols"
    ],
    "intervention_priority": "maximum"
  },
  "MODERATE_RISK": {
    "threshold": 0.40,
    "description": "Predicted success rate 40-60%",
    "recommended_actions": [
      "Standard navigation (minimum 2 contacts)",
      "Targeted barrier-specific interventions",
      "Regular follow-up"
    ],
    "intervention_priority": "standard"
  }
}
\end{lstlisting}

\subsection{Algorithm Parameters}

Technical parameters control calculation methods:

\begin{lstlisting}[language=json,caption={Algorithm parameters excerpt}]
"algorithm_parameters": {
  "barrier_combination_method": "multiplicative",
  "intervention_combination_method": "additive_with_saturation",
  "max_cumulative_intervention_effect": 0.45,
  "use_logit_space": true,
  "mechanism_diversity_weight": 0.10,
  "minimum_intervention_effect": 0.02,
  "description": "Parameters controlling probability calculations and intervention recommendations"
}
\end{lstlisting}

\section{Configuration Usage}

\subsection{Loading the Configuration}

The tool loads the configuration at runtime using Python's JSON parser:

\begin{lstlisting}[language=json,caption={Python usage example (pseudocode)}]
import json

# Load configuration
with open('lai_prep_config.json', 'r') as f:
    config = json.load(f)

# Access population data
msm_baseline = config['populations']['MSM']['baseline_attrition']
msm_interventions = config['populations']['MSM']['priority_interventions']

# Access barrier data
transport_impact = config['barriers']['TRANSPORTATION']['impact']

# Access intervention data
nav_improvement = config['interventions']['PATIENT_NAVIGATION']['improvement']
\end{lstlisting}

\subsection{Institutional Adaptation}

Institutions can modify the configuration to reflect local contexts:

\begin{enumerate}
\item \textbf{Update baseline rates}: Adjust population-specific attrition rates based on local data
\item \textbf{Modify barrier impacts}: Calibrate barrier impacts to local prevalence
\item \textbf{Add/remove interventions}: Include locally-available interventions or disable unavailable ones
\item \textbf{Adjust effect sizes}: Update intervention effects based on local implementation outcomes
\item \textbf{Set resource constraints}: Configure available interventions by healthcare setting
\end{enumerate}

\subsection{Version Control}

Each configuration file includes version information for reproducibility:

\begin{lstlisting}[language=json,caption={Version metadata}]
{
  "version": "2.0.0",
  "last_updated": "2025-01-12",
  "description": "Configuration file for LAI-PrEP Bridge Period Decision Support Tool"
}
\end{lstlisting}

\section{Evidence Integration}

All parameters are derived from peer-reviewed literature:

\begin{itemize}
\item \textbf{Baseline attrition rates}: HPTN 083, HPTN 084, PURPOSE trials, real-world cohort studies
\item \textbf{Barrier impacts}: PrEP cascade literature, implementation science studies
\item \textbf{Intervention effects}: Randomized controlled trials, systematic reviews, meta-analyses
\item \textbf{Population differences}: Clinical trial subgroup analyses, epidemiological data
\end{itemize}

\section{Patient Input Examples}

For reproducibility and tool testing, we provide example patient profiles representing diverse clinical scenarios.

\subsection{Single Patient JSON Example}

Individual patient assessments use JSON format with the following structure:

\begin{lstlisting}[language=json,caption={Example patient input (example\_patient.json)}]
{
  "patient_id": "example_001",
  "population": "PWID",
  "age": 35,
  "current_prep_status": "naive",
  "barriers": [
    "HOUSING_INSTABILITY",
    "TRANSPORTATION",
    "LEGAL_CONCERNS"
  ],
  "healthcare_setting": "HARM_REDUCTION",
  "insurance_status": "uninsured",
  "recent_hiv_test": false,
  "transportation_access": false,
  "childcare_needs": false
}
\end{lstlisting}

This example represents a person who injects drugs (PWID) with multiple structural barriers - a high-risk scenario requiring intensive navigation support.

\subsection{Batch Processing CSV Example}

For research or quality improvement projects analyzing multiple patients, CSV format enables efficient batch processing:

\begin{lstlisting}[language=json,caption={Example batch input format (example\_patients.csv)},basicstyle=\scriptsize\ttfamily]
patient_id,population,age,current_prep_status,barriers,setting,...
patient_001,MSM,28,oral_prep,SCHEDULING_CONFLICTS,LGBTQ_CENTER,...
patient_002,CISGENDER_WOMEN,32,naive,"TRANSPORTATION,CHILDCARE",...
patient_003,PWID,35,naive,"HOUSING_INSTABILITY,LEGAL_CONCERNS",...
patient_004,ADOLESCENT,17,naive,"PRIVACY_CONCERNS",...
patient_005,TRANSGENDER_WOMEN,26,discontinued_oral,DISCRIMINATION,...
\end{lstlisting}

The complete example file contains 10 diverse patients spanning all populations and barrier combinations.

\subsection{Clinical Scenarios Represented}

Example patients cover the full spectrum of bridge period challenges:

\begin{itemize}
\item \textbf{Low-barrier case}: MSM on oral PrEP, insured, same-day switching candidate
\item \textbf{Moderate-barrier case}: Cisgender woman, transportation + childcare needs
\item \textbf{High-barrier case}: PWID, housing instability + multiple barriers
\item \textbf{Special populations}: Adolescents (privacy), pregnant/lactating (competing priorities)
\item \textbf{Healthcare settings}: LGBTQ centers, harm reduction programs, community clinics
\end{itemize}

\subsection{Usage Examples}

These files enable immediate tool testing:

\begin{lstlisting}[language=json,caption={Command-line usage}]
# Assess single patient
python cli.py assess -i example_patient.json -o results.json

# Batch process multiple patients
python cli.py batch -i example_patients.csv -o batch_results.json

# Validate configuration
python cli.py validate -c lai_prep_config.json
\end{lstlisting}

\section{Data Availability}

\subsection{Complete Configuration File}

The full \texttt{lai\_prep\_config.json} file (558 lines) containing all populations, barriers, interventions, and algorithm parameters is available at:

\begin{itemize}

\item \textbf{Zenodo DOI}:\url{https://zenodo.org/uploads/17727117#:~:text=10.5281/zenodo.17727117} 
\end{itemize}

\subsection{Patient Input Examples}

Example patient files for reproducibility testing:

\begin{itemize}
\item \textbf{example\_patient.json}: Single patient JSON template with inline documentation
\item \textbf{example\_patients.csv}: Batch file with 10 diverse clinical scenarios
\item Both available in the GitHub repository under \texttt{/examples/} directory
\end{itemize}

\subsection{Supplementary Documentation}

Additional implementation materials include:

\begin{itemize}
\item \textbf{Supplementary File S1}: Clinician Quick-Reference Card
\item \textbf{Supplementary File S2}: Patient Information Handout
\item \textbf{Supplementary File S4}: Implementation Guide
\item \textbf{Supplementary File S5}: Clinical Decision Flowchart
\item \textbf{Supplementary File S6}: Non-Technical Summary
\end{itemize}

\section{Quality Assurance}

The configuration undergoes rigorous validation:

\begin{enumerate}
\item \textbf{JSON schema validation}: Ensures structural integrity
\item \textbf{Parameter range checking}: Validates all values are within reasonable bounds
\item \textbf{Reference integrity}: Confirms all intervention/barrier references are valid
\item \textbf{Evidence documentation}: Requires citation for all effect sizes
\item \textbf{Unit testing}: 18 edge cases verify correct parameter usage (100\% pass rate)
\end{enumerate}

\section{Future Updates}

The configuration is designed to evolve with emerging evidence:

\begin{itemize}
\item \textbf{New trial data}: PURPOSE-4 (PWID), HPTN 084-01 (adolescents), RUBY-4 (serodiscordant couples)
\item \textbf{Real-world implementation}: Prospective validation studies in diverse settings
\item \textbf{Cost-effectiveness data}: Economic evaluations of intervention strategies
\item \textbf{Health equity research}: Population-specific barriers and interventions
\end{itemize}

Updates will be versioned and documented with change logs to maintain reproducibility.

\vspace{1cm}

\section*{Technical Support}

For questions about the configuration file or tool implementation:

\begin{itemize}
\item \testbf{Zenodo dataset}: {\href{https://zenodo.org/uploads/17727117#:~:text=10.5281/zenodo.17727117}{LAI-PrEP Bridge Period Clinical Decision Support Tool: Algorithm, Validation Data, and Supporting Evidence for the PrEP Cascade Paradigm Shift}}
\item \textbf{Email}: acdemidont@nyxdynamics.org
\item \textbf{Documentation}: Complete API reference and integration guides in repository
\end{itemize}

\end{document}
