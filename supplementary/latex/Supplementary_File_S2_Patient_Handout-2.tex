\documentclass[11pt]{article}
\usepackage[margin=1in]{geometry}
\usepackage{helvet}
\renewcommand{\familydefault}{\sfdefault}
\usepackage{xcolor}
\usepackage{titlesec}
\usepackage{enumitem}
\usepackage{hyperref}
\usepackage{amssymb}
\usepackage{tcolorbox}

\titleformat{\section}{\Large\bfseries\color{blue!70!black}}{\thesection}{1em}{}
\titleformat{\subsection}{\large\bfseries\color{blue!50!black}}{\thesubsection}{1em}{}
\titleformat{\subsubsection}{\normalsize\bfseries\color{blue!40!black}}{\thesubsubsection}{1em}{}

\begin{document}

\begin{center}
{\Huge\bfseries Supplementary File S2}\\[0.3cm]
{\LARGE Patient Information Handout}\\[0.5cm]
{\large Your Guide to Starting Long-Acting Injectable PrEP}\\
{\large What to Expect Between Your Prescription and First Injection}\\[0.3cm]
{\normalsize AC Demidont\\[0.2cm]
{\small\textit{Viruses} Journal Supplementary Materials}
\end{center}

\vspace{0.5cm}

\section{What is Long-Acting Injectable PrEP?}

Long-acting injectable PrEP is a medication that prevents HIV infection. Instead of taking a daily pill, you get an injection every \textbf{2 months} (cabotegravir) or \textbf{6 months} (lenacapavir).

\subsection{It Works Really Well}

\begin{itemize}[leftmargin=*]
\item Over 96\% effective at preventing HIV
\item Once you start, 8 out of 10 people keep getting their injections
\item Most people prefer injections over daily pills
\end{itemize}

\section{The ``Bridge Period'' -- What You Need to Know}

\subsection{What Is It?}

The ``bridge period'' is the time \textbf{between when your doctor prescribes the injection and when you actually get your first shot}. This usually takes \textbf{2--4 weeks}, but can be shorter if you're already on PrEP pills.

\subsection{Why Can't I Start Today?}

Unlike PrEP pills (which you can start the same day), the injection lasts for months in your body. We need to make absolutely sure you don't have HIV before giving you the injection. This requires:

\begin{enumerate}[leftmargin=*]
\item $\checkmark$ An HIV test
\item $\checkmark$ Waiting for results
\item $\checkmark$ Insurance approval (if needed)
\item $\checkmark$ Scheduling your injection appointment
\end{enumerate}

\begin{tcolorbox}[colback=red!5!white,colframe=red!75!black,title=Important]
During this waiting period, you are \textbf{not protected} from HIV yet. If you're at risk, talk to your doctor about using daily PrEP pills while you wait, or other prevention methods.
\end{tcolorbox}

\section{What Happens Next? Your Timeline}

\subsection{TODAY (Prescription Visit)}

\begin{itemize}[label=$\square$,leftmargin=*]
\item Your doctor orders an HIV test
\item Your doctor submits insurance approval (if needed)
\item You schedule your HIV test appointment
\item You might be assigned a ``navigator'' to help you through the process
\end{itemize}

\subsection{Within 1--2 Days}

\begin{itemize}[label=$\square$,leftmargin=*]
\item You get your HIV test (blood draw or rapid test)
\item A navigator (if assigned) will call you to check in
\end{itemize}

\subsection{Within 3--7 Days}

\begin{itemize}[label=$\square$,leftmargin=*]
\item Your HIV test results come back
\item Your doctor reviews results with you
\item Your injection appointment is confirmed
\end{itemize}

\subsection{Within 2--4 Weeks (Goal)}

\begin{itemize}[label=$\square$,leftmargin=*]
\item \textbf{You get your first injection!}
\item You schedule your next injection (2 or 6 months away)
\end{itemize}

\section{How We'll Stay in Touch}

You'll receive:

\begin{itemize}[leftmargin=*]
\item \textbf{Text messages} reminding you about appointments
\item \textbf{Phone calls} from your navigator or clinic
\item \textbf{Email} (if you prefer)
\end{itemize}

\begin{tcolorbox}[colback=blue!5!white,colframe=blue!75!black,title=Important]
Please respond to texts and calls! We want to make sure you successfully get started.
\end{tcolorbox}

\section{What If I Need Help?}

\subsection{Transportation}

\textbf{Problem:} ``I don't have a way to get to my appointments.''

\textbf{Solution:} Tell your navigator or clinic! Many programs can provide:
\begin{itemize}[leftmargin=*]
\item Uber/Lyft vouchers
\item Public transit passes
\item Mileage reimbursement
\end{itemize}

\subsection{Childcare}

\textbf{Problem:} ``I can't leave my kids to go to appointments.''

\textbf{Solution:} Ask about:
\begin{itemize}[leftmargin=*]
\item On-site childcare at the clinic
\item Childcare vouchers
\item Home-based services
\end{itemize}

\subsection{Schedule Conflicts}

\textbf{Problem:} ``I work during clinic hours.''

\textbf{Solution:} Many clinics offer:
\begin{itemize}[leftmargin=*]
\item Early morning or evening appointments
\item Weekend hours
\item Flexible scheduling
\end{itemize}

\subsection{Privacy Concerns}

\textbf{Problem:} ``I don't want my family to know I'm taking PrEP.''

\textbf{Solution:} Tell your doctor! They can:
\begin{itemize}[leftmargin=*]
\item Use confidential communication methods
\item Help manage insurance statements
\item Protect your privacy
\end{itemize}

\subsection{Insurance Issues}

\textbf{Problem:} ``My insurance denied the medication.''

\textbf{Solution:} Your clinic can:
\begin{itemize}[leftmargin=*]
\item Appeal the denial
\item Help with patient assistance programs
\item Find alternative payment options
\end{itemize}

\subsection{Other Barriers}

Whatever barrier you're facing, \textbf{please tell your navigator or doctor}. They want to help you succeed!

\section{Red Flags -- When to Call}

Call your clinic if:

\begin{itemize}[leftmargin=*]
\item $\times$ You haven't heard from anyone within 3 days of your prescription
\item $\times$ Your HIV test wasn't scheduled
\item $\times$ It's been over a week and you don't have test results
\item $\times$ Your insurance denied coverage and no one has called you
\item $\times$ You can't make your scheduled appointments
\item $\times$ You have questions or concerns
\end{itemize}

\begin{tcolorbox}[colback=orange!5!white,colframe=orange!75!black,title=Don't Wait!]
The sooner you call, the sooner we can fix the problem!
\end{tcolorbox}

\section{Frequently Asked Questions}

\subsection{``Why is this taking so long?''}

We know it's frustrating to wait. The injection lasts for months, so we must be 100\% certain you don't have HIV. This is for your safety.

\subsection{``Can I start PrEP pills while I wait?''}

\textbf{YES!} Ask your doctor. Many people take daily PrEP pills during the bridge period to stay protected, then switch to the injection when ready.

\subsection{``What if I miss an appointment?''}

\textbf{Call us immediately!} We can reschedule. Missing one appointment doesn't mean you can't get the injection -- we just need to get you back on track.

\subsection{``Will the injection hurt?''}

Most people describe it as similar to a flu shot. The injection site might be sore for a few days. Your doctor will tell you how to manage any discomfort.

\subsection{``What happens after my first injection?''}

You'll schedule your next injection (2 or 6 months away). After the first injection, it gets much easier -- you just show up for your scheduled appointments!

\subsection{``What if I change my mind?''}

That's okay! You can stop at any time. Just let your doctor know. If you're not sure, talk to your navigator -- they can address your concerns.

\section{Tips for Success}

\subsection{\texorpdfstring{$\checkmark$}{Checkmark} Respond to texts and calls}

Even a quick ``yes'' or ``got it'' helps us know you're still on track.

\subsection{\texorpdfstring{$\checkmark$}{Checkmark} Ask for help early}

Don't wait until the day of your appointment to figure out transportation or childcare.

\subsection{\texorpdfstring{$\checkmark$}{Checkmark} Keep your appointments}

Every appointment gets you closer to starting. If you can't make it, call to reschedule right away.

\subsection{\texorpdfstring{$\checkmark$}{Checkmark} Be honest about barriers}

We can't help if we don't know what challenges you're facing.

\subsection{\texorpdfstring{$\checkmark$}{Checkmark} Stay motivated}

Remember why you wanted the injection! Once you get started, you won't have to think about daily pills anymore.

\section{Your Team}

You're not alone in this process. Your team includes:

\subsection{Your Doctor/Nurse Practitioner}
\begin{itemize}[leftmargin=*]
\item Prescribed the injection
\item Reviews your HIV tests
\item Administers your injections
\end{itemize}

\subsection{Your Navigator (if assigned)}
\begin{itemize}[leftmargin=*]
\item Helps you overcome barriers
\item Coordinates appointments
\item Answers questions
\item Available by phone/text
\end{itemize}

\subsection{Clinic Staff}
\begin{itemize}[leftmargin=*]
\item Schedules appointments
\item Processes insurance
\item Sends reminders
\end{itemize}

\subsection{YOU}
\begin{itemize}[leftmargin=*]
\item The most important member!
\item Stay in communication
\item Ask for help when needed
\item Show up for appointments
\end{itemize}

\section{Contact Information}

\begin{tabular}{ll}
\textbf{Your Navigator:} & \underline{\hspace{6cm}} \\[0.3cm]
\textbf{Phone:} & \underline{\hspace{6cm}} \\[0.3cm]
\textbf{Text:} & \underline{\hspace{6cm}} \\[0.3cm]
\textbf{Clinic Main Number:} & \underline{\hspace{6cm}} \\[0.3cm]
\textbf{Emergency/Urgent Questions:} & \underline{\hspace{6cm}} \\[0.3cm]
\textbf{Best Time to Reach You:} & \underline{\hspace{6cm}} \\[0.3cm]
\textbf{Preferred Contact Method:} & Phone / Text / Email (circle one) \\
\end{tabular}

\section{Why This Matters}

HIV prevention is serious healthcare. Research shows that almost \textbf{half of people who are prescribed this injection never get their first shot}. Not because the medication doesn't work -- it works great! But because of the challenges in the weeks between prescription and injection.

\textbf{That's why we're putting in this extra effort to help you succeed.}

The injection is highly effective and convenient once you start. We just need to get you through these first few weeks. With your participation and our support, you WILL get there!

\section{Quick Checklist -- Keep This Handy}

\subsection{Before your first injection, you need:}

\begin{itemize}[label=$\square$,leftmargin=*]
\item HIV test completed
\item Results reviewed (must be negative)
\item Insurance approved (or payment arranged)
\item Injection appointment scheduled
\item Transportation arranged
\item Childcare arranged (if needed)
\item Questions answered
\end{itemize}

\subsection{When you're ready:}

\begin{itemize}[label=$\square$,leftmargin=*]
\item Show up for your appointment
\item Get your injection
\item Schedule next appointment
\item Done! Much easier from here on!
\end{itemize}

\section{Remember}

\begin{itemize}[leftmargin=*]
\item \textbf{96\% effective} once you start
\item \textbf{4--6 appointments per year} (vs.\ 365 daily pills)
\item \textbf{Most people prefer} injections over pills
\item \textbf{Your navigator is here to help} -- use them!
\item \textbf{Thousands of people} have successfully started -- you can too!
\end{itemize}

\section{After Your First Injection}

Once you get your first injection, the hard part is over! You'll come back in \textbf{2 months} (cabotegravir) or \textbf{6 months} (lenacapavir) for your next shot. No daily pills to remember. No monthly pharmacy visits.

\begin{center}
\textbf{\Large You've got this!}
\end{center}

\vspace{0.5cm}

\begin{tcolorbox}[colback=gray!5!white,colframe=gray!75!black]
\textit{This guide is based on research published by Demidont \&  (2025) in Viruses journal. The ``bridge period'' is a newly recognized challenge in LAI-PrEP implementation, and your clinic is using evidence-based strategies to help you succeed.}

\textbf{Questions? Call your navigator or clinic -- we're here to help!}
\end{tcolorbox}

\section{For More Information}

\begin{itemize}[leftmargin=*]
\item \textbf{CDC PrEP Information:} \url{www.cdc.gov/hiv/basics/prep.html}
\item \textbf{PrEP Hotline} (free, confidential): 1-855-HIV-PrEP (1-855-448-7737)
\item \textbf{National Clinician Consultation Center:} \url{nccc.ucsf.edu}
\item \textbf{PrEPWatch} (LAI-PrEP information): \url{www.prepwatch.org}
\end{itemize}

\end{document}
