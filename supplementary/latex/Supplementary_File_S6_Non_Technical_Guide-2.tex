\documentclass[11pt]{article}
\usepackage[margin=1in]{geometry}
\usepackage{helvet}
\renewcommand{\familydefault}{\sfdefault}
\usepackage{xcolor}
\usepackage{titlesec}
\usepackage{enumitem}
\usepackage{hyperref}
\usepackage{graphicx}
\usepackage{booktabs}
\usepackage{longtable}
\usepackage{array}
\usepackage{tcolorbox}
\usepackage{amssymb}

\titleformat{\section}{\Large\bfseries\color{blue!70!black}}{\thesection}{1em}{}
\titleformat{\subsection}{\large\bfseries\color{blue!50!black}}{\thesubsection}{1em}{}
\titleformat{\subsubsection}{\normalsize\bfseries\color{blue!40!black}}{\thesubsubsection}{1em}{}

% Define color boxes for key information
\newtcolorbox{clinicaltip}{
  colback=blue!5!white,
  colframe=blue!75!black,
  fonttitle=\bfseries,
  title=Clinical Tip
}

\newtcolorbox{keypoint}{
  colback=green!5!white,
  colframe=green!65!black,
  fonttitle=\bfseries,
  title=Key Takeaway
}

\newtcolorbox{warning}{
  colback=orange!5!white,
  colframe=orange!75!black,
  fonttitle=\bfseries,
  title=Important Note
}

\begin{document}

\begin{center}
{\Huge\bfseries Supplementary File S6}\\[0.3cm]
{\LARGE LAI-PrEP Bridge Period Decision Tool}\\[0.2cm]
{\Large A Non-Technical Guide for Clinicians and Healthcare Workers}\\[0.5cm]
{\normalsize A.C Demidont, DO}\\[0.2cm]
{\small\textit{Viruses} Journal Supplementary Materials}\\[0.2cm]
{\footnotesize Accompanying: ``Bridging the Gap: The PrEP Cascade Paradigm Shift for Long-Acting Injectable HIV Prevention''}
\end{center}

\vspace{0.5cm}

\tableofcontents
\newpage

\section{What Is This Tool?}

This is a \textbf{clinical decision support calculator} that helps you predict whether a patient will successfully complete the ``bridge period'' (the time between prescribing LAI-PrEP and administering the first injection).

Think of it like a \textbf{risk calculator} similar to cardiovascular risk scores (like Framingham) or fracture risk tools (like FRAX), but specifically designed for LAI-PrEP implementation.

\subsection{Why This Matters}

Research shows that \textbf{only 53\% of patients who are prescribed LAI-PrEP actually receive their first injection}. The other 47\% are lost during the bridge period due to various barriers. This tool helps you:

\begin{enumerate}[leftmargin=*]
\item \textbf{Predict} which patients are at high risk of never starting
\item \textbf{Identify} specific barriers preventing initiation
\item \textbf{Select} interventions proven to improve success rates
\item \textbf{Estimate} how much those interventions will help
\end{enumerate}

\begin{keypoint}
LAI-PrEP demonstrates superior clinical outcomes (96\% efficacy, 81--83\% persistence), but only if patients successfully initiate treatment. The bridge period is where implementation fails.
\end{keypoint}

\section{Understanding the Bridge Period}

\subsection{What is the Bridge Period?}

The \textbf{bridge period} is everything that happens between when you decide to prescribe LAI-PrEP and when the patient receives their first injection. This includes:

\begin{itemize}[leftmargin=*]
\item Getting HIV test results back (must confirm HIV-negative)
\item Scheduling the injection appointment
\item Getting insurance authorization
\item Patient arranging transportation
\item Patient managing childcare (if needed)
\item Waiting period for test results
\end{itemize}

\textbf{Duration:} Usually 2--8 weeks, depending on circumstances

\subsection{Why Does This Matter?}

Unlike oral PrEP (where patients can start the same day they get their prescription), LAI-PrEP \textbf{cannot} be started immediately. This delay creates opportunities for patients to:

\begin{itemize}[leftmargin=*]
\item Change their mind
\item Face logistical barriers
\item Lose motivation
\item Get frustrated with the process
\item Acquire HIV during the waiting period
\end{itemize}

\begin{warning}
\textbf{47\% of patients never make it through this bridge period.} The tool helps you prevent that by identifying high-risk patients and implementing targeted interventions.
\end{warning}

\section{How Does the Tool Work?}

\subsection{Step 1: Input Patient Information}

The tool asks for basic information about your patient:

\subsubsection{Population Category}

Which group best describes your patient?
\begin{itemize}[leftmargin=*]
\item Men who have sex with men (MSM)
\item Cisgender women
\item Transgender women
\item Adolescents (ages 16--24)
\item People who inject drugs (PWID)
\item Pregnant or lactating individuals
\item General population
\end{itemize}

\begin{clinicaltip}
\textbf{Why this matters:} Different populations face different barriers. For example, adolescents have a 60--70\% risk of not completing the bridge period, while MSM have a 40--50\% risk.
\end{clinicaltip}

\subsubsection{Current PrEP Status}

\begin{itemize}[leftmargin=*]
\item \textbf{Never been on PrEP} (``naive'') -- Highest risk
\item \textbf{Currently taking oral PrEP} -- Much lower risk (85--90\% success!)
\item \textbf{Previously on oral PrEP but stopped} -- Moderate risk
\end{itemize}

\begin{keypoint}
Patients already on oral PrEP are your BEST candidates -- they've already proven they can navigate the healthcare system and are motivated for prevention.
\end{keypoint}

\subsubsection{Recent HIV Test?}

\begin{itemize}[leftmargin=*]
\item Has the patient had an HIV test within the last 7 days?
\item YES = Can potentially start same day or very soon
\item NO = Will need testing, adding 2--3 weeks to bridge period
\end{itemize}

\subsubsection{Barriers Present}

Does your patient face any of these challenges? \textbf{Check ALL that apply} -- each barrier increases risk.

\textit{Access Barriers:}
\begin{itemize}[leftmargin=*]
\item[$\square$] Transportation difficulties
\item[$\square$] Childcare needs
\item[$\square$] Housing instability
\item[$\square$] Insurance authorization delays expected
\item[$\square$] Scheduling conflicts (work, school)
\item[$\square$] No government ID
\end{itemize}

\textit{Trust and Safety Barriers:}
\begin{itemize}[leftmargin=*]
\item[$\square$] Medical mistrust
\item[$\square$] Privacy/confidentiality concerns
\item[$\square$] Past healthcare discrimination
\item[$\square$] Legal concerns (for PWID)
\end{itemize}

\textit{System Navigation Barriers:}
\begin{itemize}[leftmargin=*]
\item[$\square$] Competing health priorities
\item[$\square$] Limited experience navigating healthcare
\item[$\square$] Active substance use
\end{itemize}

\subsubsection{Healthcare Setting}

Where will the patient receive care?
\begin{itemize}[leftmargin=*]
\item Academic medical center
\item Community health center
\item Private practice
\item Pharmacy
\item Harm reduction/syringe service program
\item LGBTQ community center
\item Mobile clinic
\item Telehealth-integrated
\end{itemize}

\subsection{Step 2: Tool Calculates Risk}

The tool automatically calculates:

\begin{enumerate}[leftmargin=*]
\item \textbf{Baseline Success Rate:} Starting point based on population (e.g., MSM = 55\%)
\item \textbf{Adjusted Success Rate:} After accounting for individual barriers (may drop to 20\% or lower with multiple barriers)
\item \textbf{Risk Level:} Low, Moderate, High, or Very High
\item \textbf{Estimated Bridge Duration:} How long from prescription to injection (0--56 days)
\end{enumerate}

\subsection{Step 3: Tool Recommends Interventions}

The tool provides a \textbf{prioritized list} of interventions, ranked by:
\begin{itemize}[leftmargin=*]
\item \textbf{Priority Level:} Critical, High, or Moderate
\item \textbf{Expected Improvement:} How many percentage points this intervention adds to success rate
\item \textbf{Evidence Level:} Strong, Moderate, or Emerging
\end{itemize}

\textit{Example output:}
\begin{verbatim}
1. Same-day switching protocol
   Priority: CRITICAL
   Expected Improvement: +40 percentage points
   Evidence: Strong
   
2. Patient navigation program
   Priority: High
   Expected Improvement: +15 percentage points
   Evidence: Strong
\end{verbatim}

\subsection{Step 4: Tool Predicts Final Outcome}

The tool estimates success rate \textbf{WITH} your top interventions implemented:
\begin{center}
Baseline: 20\% $\rightarrow$ With interventions: 48\% (+28 points)
\end{center}

This helps you understand if your interventions are sufficient or if more intensive support is needed.

\section{Real-World Examples}

\subsection{Example 1: Best Case Scenario}

\textbf{Patient:} 28-year-old MSM currently on oral PrEP, had HIV test 3 days ago

\textbf{Tool Output:}
\begin{itemize}[leftmargin=*]
\item Baseline Success: 55\%
\item With barriers: 50\% (mild scheduling conflict)
\item \textbf{Recommended Action:} Same-day switching protocol
\item \textbf{Final Success Prediction:} 86\%
\item \textbf{Bridge Duration:} 0--3 days
\end{itemize}

\begin{keypoint}
\textbf{What This Means:} This patient is a PRIORITY for immediate transition. Don't make them wait! You can inject today or within a few days. This is your easiest case.
\end{keypoint}

\textbf{Action Steps:}
\begin{enumerate}[leftmargin=*]
\item Confirm HIV test is current (\checkmark -- 3 days ago)
\item Schedule injection appointment for this week
\item Submit insurance authorization immediately
\item Done! Patient has 86\% chance of success
\end{enumerate}

\subsection{Example 2: Moderate Risk Case}

\textbf{Patient:} 32-year-old cisgender woman who stopped oral PrEP 6 months ago, has transportation and childcare barriers

\textbf{Tool Output:}
\begin{itemize}[leftmargin=*]
\item Baseline Success: 45\%
\item With barriers: 17\% (very high risk!)
\item \textbf{Recommended Actions:}
\begin{enumerate}
  \item Patient navigation program (+15 points)
  \item Transportation vouchers (+8 points)
  \item Childcare support (+8 points)
\end{enumerate}
\item \textbf{Final Success Prediction:} 45\%
\item \textbf{Bridge Duration:} 21--56 days
\end{itemize}

\begin{warning}
Without help, this patient has only a 17\% chance of starting LAI-PrEP. But with navigation, transportation, and childcare support, you can increase success to 45\%.
\end{warning}

\textbf{Action Steps:}
\begin{enumerate}[leftmargin=*]
\item Assign patient navigator immediately
\item Provide Uber/Lyft vouchers for appointments
\item Offer on-site childcare or childcare vouchers
\item Navigator should call within 24 hours of prescription
\item Schedule HIV testing at most convenient location
\item Text reminders 48h and 24h before appointments
\end{enumerate}

\subsection{Example 3: Very High Risk Case}

\textbf{Patient:} 35-year-old person who injects drugs (PWID), experiencing housing instability, no ID, multiple barriers

\textbf{Tool Output:}
\begin{itemize}[leftmargin=*]
\item Baseline Success: 25\%
\item With barriers: 5\% (extremely high risk!)
\item \textbf{Recommended Actions:}
\begin{enumerate}
  \item Harm reduction integration (SSP) (+15 points)
  \item Peer navigation (+15 points)
  \item Mobile delivery (+12 points)
  \item Accelerated testing (+10 points)
\end{enumerate}
\item \textbf{Final Success Prediction:} 31\%
\item \textbf{Bridge Duration:} 21--56 days
\end{itemize}

\begin{warning}
Traditional clinic-based approach will fail. This patient needs services brought to them in a trusted setting with peer support.
\end{warning}

\textbf{Action Steps:}
\begin{enumerate}[leftmargin=*]
\item Partner with local syringe service program (SSP)
\item Assign peer navigator with lived experience
\item Arrange mobile testing at SSP location
\item Use point-of-care HIV testing (results in 20 minutes)
\item Schedule injection at SSP site or mobile clinic
\item Eliminate ID requirements through low-barrier protocols
\item Provide injection at most convenient time/location for patient
\item Build trust through non-judgmental, harm reduction-informed care
\end{enumerate}

\section{Quick Reference Tables}

\subsection{Table 1: Population-Specific Baseline Success Rates}

\begin{table}[h]
\centering
\begin{tabular}{lcc}
\toprule
\textbf{Population} & \textbf{Baseline Success} & \textbf{Baseline Attrition} \\
\midrule
MSM & 55\% & 45\% \\
Cisgender Women & 45\% & 55\% \\
Transgender Women & 40\% & 60\% \\
Adolescents (16--24) & 35\% & 65\% \\
PWID & 25\% & 75\% \\
Pregnant/Lactating & 50\% & 50\% \\
General Population & 50\% & 50\% \\
\bottomrule
\end{tabular}
\caption{Baseline initiation success rates by population, derived from clinical trial data (HPTN 083, 084, PURPOSE-1/2) and implementation studies.}
\end{table}

\subsection{Table 2: Barrier Impact on Success Rate}

\begin{table}[h]
\centering
\small
\begin{tabular}{lc}
\toprule
\textbf{Barrier} & \textbf{Impact on Success} \\
\midrule
Transportation difficulties & -12\% \\
Childcare needs & -10\% \\
Housing instability & -15\% \\
Insurance delays & -8\% \\
Scheduling conflicts & -5\% \\
Medical mistrust & -12\% \\
Privacy concerns & -8\% \\
Past discrimination & -10\% \\
Competing health priorities & -7\% \\
Limited healthcare navigation & -10\% \\
Legal concerns (PWID) & -15\% \\
No government ID & -12\% \\
Active substance use & -10\% \\
\bottomrule
\end{tabular}
\caption{Quantified impact of structural barriers on bridge period success rate. Impacts are cumulative but not strictly additive due to overlapping mechanisms.}
\end{table}

\subsection{Table 3: Evidence-Based Interventions}

\begin{longtable}{lcp{6cm}}
\toprule
\textbf{Intervention} & \textbf{Improvement} & \textbf{Best For} \\
\midrule
\endfirsthead
\multicolumn{3}{c}{\textit{(continued from previous page)}} \\
\toprule
\textbf{Intervention} & \textbf{Improvement} & \textbf{Best For} \\
\midrule
\endhead
\midrule
\multicolumn{3}{r}{\textit{(continued on next page)}} \\
\endfoot
\bottomrule
\endlastfoot
Same-day switching & +40\% & Patient on oral PrEP + recent HIV test \\
Oral-to-injectable transition & +35\% & Patient on oral PrEP (no recent test) \\
Patient navigation & +15\% & Any high-risk patient or vulnerable population \\
Harm reduction integration & +15\% & PWID -- ESSENTIAL \\
Peer navigation & +12\% & PWID, adolescents, transgender individuals \\
Accelerated HIV testing & +10\% & All PrEP-naive patients \\
Transportation support & +8\% & When transportation is a known barrier \\
Childcare support & +8\% & Parents with childcare responsibilities \\
Medical mistrust intervention & +10\% & Populations with healthcare mistrust \\
Anti-discrimination protocols & +12\% & LGBTQ+ populations, PWID \\
Confidentiality protections & +8\% & Adolescents, populations with privacy concerns \\
Flexible scheduling & +6\% & Patients with work/school conflicts \\
Low-barrier protocols & +12\% & PWID, unhoused individuals \\
Pregnancy counseling & +8\% & Pregnant individuals \\
Prenatal integration & +10\% & Pregnant individuals \\
Insurance support & +10\% & All patients with insurance barriers \\
Mobile delivery & +12\% & Hard-to-reach populations \\
\caption{Evidence-based interventions with quantified improvements in bridge period success rates. Evidence levels range from strong (clinical trials, systematic reviews) to moderate (implementation studies).}
\end{longtable}

\subsection{Table 4: Bridge Period Duration}

\begin{table}[h]
\centering
\begin{tabular}{lcp{6cm}}
\toprule
\textbf{Scenario} & \textbf{Duration} & \textbf{Notes} \\
\midrule
Oral PrEP + recent test & 0--3 days & Same-day possible \\
Oral PrEP + need test & 7--14 days & Fast track \\
PrEP-naive + minimal barriers & 14--35 days & Standard \\
PrEP-naive + multiple barriers & 35--56 days & High attrition risk \\
\bottomrule
\end{tabular}
\caption{Typical bridge period durations by patient scenario.}
\end{table}

\section{Setting Up Your Program}

\subsection{What You Need to Implement This}

\subsubsection{Minimal Setup (For Low-Risk Patients)}

\begin{itemize}[leftmargin=*]
\item[$\checkmark$] Protocol for same-day switching (oral PrEP patients)
\item[$\checkmark$] Rapid HIV testing turnaround ($<$ 48 hours)
\item[$\checkmark$] Text message reminder system
\item[$\checkmark$] Staff training on LAI-PrEP basics
\end{itemize}

\subsubsection{Standard Setup (For Moderate-Risk Patients)}

Everything above, PLUS:
\begin{itemize}[leftmargin=*]
\item[$\checkmark$] Patient navigator (can be part-time)
\item[$\checkmark$] Transportation voucher program
\item[$\checkmark$] Expedited insurance authorization process
\item[$\checkmark$] Telehealth capability for counseling
\end{itemize}

\subsubsection{Comprehensive Setup (For High-Risk Populations)}

Everything above, PLUS:
\begin{itemize}[leftmargin=*]
\item[$\checkmark$] Full-time dedicated navigator
\item[$\checkmark$] Childcare support or on-site childcare
\item[$\checkmark$] Partnership with harm reduction services
\item[$\checkmark$] Peer navigators for key populations
\item[$\checkmark$] Mobile delivery capability
\item[$\checkmark$] Flexible scheduling (evenings/weekends)
\end{itemize}

\subsection{Staffing Models}

\textbf{Option 1: Nurse Navigator}
\begin{itemize}[leftmargin=*]
\item Best for academic medical centers and large community health centers
\item Can handle clinical tasks (testing, injection)
\item Typical caseload: 100--150 patients
\end{itemize}

\textbf{Option 2: Pharmacist Navigator}
\begin{itemize}[leftmargin=*]
\item Ideal for pharmacy-based programs
\item Can prescribe (in states with authority)
\item Extended hours (evenings, weekends)
\end{itemize}

\textbf{Option 3: Community Health Worker}
\begin{itemize}[leftmargin=*]
\item Best for underserved populations
\item Culturally concordant support
\item Can assist with non-clinical barriers
\end{itemize}

\textbf{Option 4: Peer Navigator}
\begin{itemize}[leftmargin=*]
\item ESSENTIAL for PWID populations
\item Highly effective for LGBTQ+ populations
\item Lived experience builds trust
\end{itemize}

\textbf{Option 5: Hybrid Model}
\begin{itemize}[leftmargin=*]
\item Nurse/pharmacist for clinical tasks
\item Peer/CHW for barrier navigation
\item Most comprehensive but most resource-intensive
\end{itemize}

\section{Frequently Asked Questions}

\subsection{Do I need to know how to code to use this?}

\textbf{No!} This guide provides all the information you need without touching code. You can:
\begin{itemize}[leftmargin=*]
\item Use the reference tables to quickly assess risk
\item Follow the decision trees for intervention selection
\item Apply the principles in your clinical practice
\end{itemize}

If your organization wants to implement the actual computer tool, your IT department can help set it up.

\subsection{How accurate is this tool?}

The tool is based on published data from:
\begin{itemize}[leftmargin=*]
\item \textbf{Over 15,000 participants} in clinical trials (HPTN 083, 084, PURPOSE-1, PURPOSE-2)
\item \textbf{Real-world implementation studies} (CAN Community Health Network Study)
\item \textbf{Systematic reviews} of patient navigation in healthcare
\end{itemize}

Predictions are \textbf{population-level estimates} -- individual patients may vary, but the tool provides scientifically-grounded guidance.

\subsection{What if my patient doesn't fit neatly into one category?}

Use your clinical judgment to select the \textbf{closest match}. For example:
\begin{itemize}[leftmargin=*]
\item A 25-year-old MSM could be categorized as either ``MSM'' or ``Adolescent'' -- choose based on which barriers seem most relevant
\item A transgender man who has sex with men might be best categorized as ``MSM'' for risk prediction purposes
\end{itemize}

The barriers list is more important than perfect category matching.

\subsection{Can I use this tool for lenacapavir AND cabotegravir?}

\textbf{Yes!} The bridge period challenges apply to both formulations. The main differences:
\begin{itemize}[leftmargin=*]
\item Lenacapavir: Every 6 months (fewer appointments)
\item Cabotegravir: Every 2 months (more frequent)
\item Lenacapavir is subcutaneous (under skin), cabotegravir is intramuscular (into muscle)
\end{itemize}

The tool's predictions apply to both.

\subsection{What about once-yearly lenacapavir?}

Once-yearly formulations are currently in Phase 3 trials (expected results second half of 2025). When approved, the bridge period will likely be:
\begin{itemize}[leftmargin=*]
\item \textbf{Longer} (more conservative testing needed due to year-long exposure)
\item \textbf{More crucial} (missing one injection = entire year without protection)
\end{itemize}

The tool's principles will apply, but specific numbers may need updating.

\subsection{How do I know if interventions are working?}

\textbf{Track these metrics:}
\begin{enumerate}[leftmargin=*]
\item \textbf{Initiation Rate:} \% of prescriptions resulting in first injection
\item \textbf{Bridge Duration:} Days from prescription to injection
\item \textbf{Attrition Reasons:} Why did patients not initiate? (insurance? transportation? lost to follow-up?)
\item \textbf{Population-Specific Rates:} Are outcomes equitable across groups?
\end{enumerate}

\textbf{Target benchmarks:}
\begin{itemize}[leftmargin=*]
\item Overall initiation rate: $>$70\% (currently only 53\% nationally)
\item Bridge duration: $<$14 days for oral-to-injectable, $<$28 days for PrEP-naive
\item Attrition due to system barriers (insurance, scheduling): $<$10\%
\end{itemize}

\subsection{This seems like a lot of work. Is it worth it?}

Consider:
\begin{itemize}[leftmargin=*]
\item \textbf{Current situation:} 47\% of patients never get their first injection -- that's a complete prevention failure
\item \textbf{LAI-PrEP advantages:} Once initiated, 81--83\% stay on LAI-PrEP (vs. only 52\% on oral PrEP)
\item \textbf{Math:} Investing in successful initiation means you DON'T have to invest in ongoing retention support
\end{itemize}

\begin{keypoint}
\textbf{It's actually LESS work overall} to get people started on LAI-PrEP successfully than to support ongoing oral PrEP adherence.
\end{keypoint}

\section{Action Steps for Your Clinic}

\subsection{This Week}

\begin{itemize}[leftmargin=*]
\item[$\square$] \textbf{Identify your current oral PrEP patients} -- they are your EASIEST wins
\item[$\square$] \textbf{Start conversations} about switching to LAI-PrEP
\item[$\square$] \textbf{Implement same-day switching} for patients with recent HIV tests
\item[$\square$] \textbf{Map your barriers} -- what do YOUR patients face?
\end{itemize}

\subsection{This Month}

\begin{itemize}[leftmargin=*]
\item[$\square$] \textbf{Designate a bridge period navigator} (even part-time)
\item[$\square$] \textbf{Set up text message reminders} for appointments
\item[$\square$] \textbf{Establish rapid HIV testing protocol} ($<$ 48 hour turnaround)
\item[$\square$] \textbf{Create transportation voucher program} (even small scale)
\item[$\square$] \textbf{Train staff} on LAI-PrEP bridge period challenges
\end{itemize}

\subsection{This Quarter}

\begin{itemize}[leftmargin=*]
\item[$\square$] \textbf{Measure your initiation rate} -- track prescriptions vs. injections
\item[$\square$] \textbf{Analyze attrition reasons} -- where are you losing patients?
\item[$\square$] \textbf{Implement population-specific interventions} based on your patient mix
\item[$\square$] \textbf{Establish community partnerships} (SSPs, LGBTQ centers, mobile clinics)
\item[$\square$] \textbf{Evaluate and adjust} your protocols
\end{itemize}

\section{Additional Resources}

\subsection{Clinical Guidelines}
\begin{itemize}[leftmargin=*]
\item \textbf{CDC:} US Public Health Service PrEP Guidelines (2021 Update)
\item \textbf{WHO:} Consolidated Guidelines on HIV Prevention (with July 2025 LAI-PrEP addendum)
\end{itemize}

\subsection{Implementation Support}
\begin{itemize}[leftmargin=*]
\item \textbf{National Clinician Consultation Center:} nccc.ucsf.edu (PrEP Quick Guide)
\item \textbf{PrEPWatch:} prepwatch.org (tracking LAI-PrEP access and implementation)
\end{itemize}

\subsection{Training}
\begin{itemize}[leftmargin=*]
\item \textbf{Clinician Consultation Center:} Free phone consultation for complex cases
\item \textbf{AETC National Coordinating Resource Center:} HIV education and training
\end{itemize}

\section{Summary: Key Takeaways}

\begin{tcolorbox}[colback=blue!5!white,colframe=blue!75!black,title=\textbf{The Core Problem}]
Only 53\% of patients prescribed LAI-PrEP actually get their first injection. The bridge period is where we lose people.
\end{tcolorbox}

\begin{tcolorbox}[colback=green!5!white,colframe=green!65!black,title=\textbf{The Core Solution}]
Proactively identify high-risk patients and implement evidence-based interventions \textbf{before} they get lost.
\end{tcolorbox}

\begin{tcolorbox}[colback=yellow!10!white,colframe=orange!75!black,title=\textbf{The Biggest Win}]
Patients already on oral PrEP have 85--90\% success with transitions. \textbf{Prioritize them!}
\end{tcolorbox}

\begin{tcolorbox}[colback=purple!5!white,colframe=purple!75!black,title=\textbf{The Equity Imperative}]
Without intentional intervention, populations most likely to benefit (adolescents, women, PWID) will have the lowest access. This is a \textbf{health equity issue}.
\end{tcolorbox}

\begin{tcolorbox}[colback=gray!5!white,colframe=gray!75!black,title=\textbf{The Evidence Base}]
This isn't guesswork -- it's based on data from over 15,000 clinical trial participants and real-world implementation studies.
\end{tcolorbox}

\vspace{1cm}

\begin{center}
\begin{tcolorbox}[width=0.9\textwidth,colback=blue!10!white,colframe=blue!75!black]
\textbf{Remember:} LAI-PrEP is clinically extraordinary ($>$96\% efficacy, 81--83\% persistence). The challenge isn't the medication -- it's the bridge period. This tool helps you bridge that gap.
\end{tcolorbox}
\end{center}

\vspace{1cm}

\begin{center}
{\small\textit{Based on: Demidont, A.C.; Backus, K.V. (2025). Bridging the Gap: The PrEP Cascade Paradigm Shift for Long-Acting Injectable HIV Prevention. Viruses.}}
\end{center}

\end{document}
